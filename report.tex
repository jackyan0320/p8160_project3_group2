\PassOptionsToPackage{unicode=true}{hyperref} % options for packages loaded elsewhere
\PassOptionsToPackage{hyphens}{url}
%
\documentclass[
]{article}
\usepackage{lmodern}
\usepackage{amssymb,amsmath}
\usepackage{ifxetex,ifluatex}
\ifnum 0\ifxetex 1\fi\ifluatex 1\fi=0 % if pdftex
  \usepackage[T1]{fontenc}
  \usepackage[utf8]{inputenc}
  \usepackage{textcomp} % provides euro and other symbols
\else % if luatex or xelatex
  \usepackage{unicode-math}
  \defaultfontfeatures{Scale=MatchLowercase}
  \defaultfontfeatures[\rmfamily]{Ligatures=TeX,Scale=1}
\fi
% use upquote if available, for straight quotes in verbatim environments
\IfFileExists{upquote.sty}{\usepackage{upquote}}{}
\IfFileExists{microtype.sty}{% use microtype if available
  \usepackage[]{microtype}
  \UseMicrotypeSet[protrusion]{basicmath} % disable protrusion for tt fonts
}{}
\makeatletter
\@ifundefined{KOMAClassName}{% if non-KOMA class
  \IfFileExists{parskip.sty}{%
    \usepackage{parskip}
  }{% else
    \setlength{\parindent}{0pt}
    \setlength{\parskip}{6pt plus 2pt minus 1pt}}
}{% if KOMA class
  \KOMAoptions{parskip=half}}
\makeatother
\usepackage{xcolor}
\IfFileExists{xurl.sty}{\usepackage{xurl}}{} % add URL line breaks if available
\IfFileExists{bookmark.sty}{\usepackage{bookmark}}{\usepackage{hyperref}}
\hypersetup{
  pdftitle={Project 3: Analyses of daily COVID-19 cases across nations},
  pdfauthor={Jiayi Shen (js5354), Siquan Wang (sw3442), Jack Yan (xy2395)},
  pdfborder={0 0 0},
  breaklinks=true}
\urlstyle{same}  % don't use monospace font for urls
\usepackage[margin=1in]{geometry}
\usepackage{graphicx,grffile}
\makeatletter
\def\maxwidth{\ifdim\Gin@nat@width>\linewidth\linewidth\else\Gin@nat@width\fi}
\def\maxheight{\ifdim\Gin@nat@height>\textheight\textheight\else\Gin@nat@height\fi}
\makeatother
% Scale images if necessary, so that they will not overflow the page
% margins by default, and it is still possible to overwrite the defaults
% using explicit options in \includegraphics[width, height, ...]{}
\setkeys{Gin}{width=\maxwidth,height=\maxheight,keepaspectratio}
\setlength{\emergencystretch}{3em}  % prevent overfull lines
\providecommand{\tightlist}{%
  \setlength{\itemsep}{0pt}\setlength{\parskip}{0pt}}
\setcounter{secnumdepth}{-2}
% Redefines (sub)paragraphs to behave more like sections
\ifx\paragraph\undefined\else
  \let\oldparagraph\paragraph
  \renewcommand{\paragraph}[1]{\oldparagraph{#1}\mbox{}}
\fi
\ifx\subparagraph\undefined\else
  \let\oldsubparagraph\subparagraph
  \renewcommand{\subparagraph}[1]{\oldsubparagraph{#1}\mbox{}}
\fi

% set default figure placement to htbp
\makeatletter
\def\fps@figure{htbp}
\makeatother


\title{Project 3: Analyses of daily COVID-19 cases across nations}
\author{Jiayi Shen (js5354), Siquan Wang (sw3442), Jack Yan (xy2395)}
\date{5/1/2020}

\begin{document}
\maketitle

\hypertarget{introduction}{%
\section{1. Introduction}\label{introduction}}

The pandemic of COVID-19 is the biggest challenge that the world is
facing right now. Our lifes are all deeply affected by this public
health crisis. By building a model on the growth of COVID-19 cases, we
can have a better understanding of the current status and then plan
future responses. Thus analyzing existing data and predicting future
trajectories has become the most important task faced by public health
expertises and policy makers.

The objective of this project includings:\\
- Develop an optmization algorithm to fit a logisitc curve to each
region, using the global COVID-19 data;\\
- Evaluate how appropiate it is to use such model;\\
- Apply clustering methods to the logistic grwoth model parameters, and
observe the patterns.

\hypertarget{data}{%
\subsection{2. Data}\label{data}}

The datasets used in this project are adapted from
\url{https://github.com/CSSEGISandData/COVID-19/tree/master/csse_covid_19_data/csse_covid_19_daily_reports}.
We used the latest data updated on 04/29/2020, so that we have more data
points. The data recorded the following variables:

\textbf{Id:} Record ID

\textbf{Province/State:} The lcoal state/province of the record; 54\%
records do not have this info;

\textbf{Country/Region:} The country/regionoof the record;

\textbf{Lat:} Lattudiute of the record;

\textbf{Long:} Longitude of the record;

\textbf{Date:} Date of the record; from Jan 21 to March 23;

\textbf{ConfirmedCases:} The number of confirme case on that day;

\textbf{Fatalities:} The number of death on that day;

For the purpose of fitting logistic growth curve, the main variables of
interest are \texttt{Country/Region} and \texttt{ConfirmedCases}. We
groupped the dataset by \texttt{Country/Region}, and pulled out
\texttt{days\ since\ first\ case} and
\texttt{cumulative\ confirmed\ cases\ on\ each\ day}.

\hypertarget{method}{%
\section{3. Method}\label{method}}

\hypertarget{logisitic-curves}{%
\subsection{3.1 Logisitic curves}\label{logisitic-curves}}

Logisitic curves could be one way to model the trajectory of cumulative
cases; It is a parametric function with the form
\[f(t) = \frac{a}{ 1+ exp\{-b(t-c)\}},\] where \(t\) is the days since
the first infection; \(a\) is the upper bound, i.e.~the maxium number of
cases a region can reach, \(b\) is growth rate, and \(c\) is the
mid-point, where the curve changes from convex to concave; Each curve is
uniquely defined by \((a,b,c)\). By design a logistic curve increases
exponentially at begining and slows down at the end.

Consider the most commonly used loss function: Mean Squared Error (MSE),
which takes the form \[
l = \frac{1}{n} \sum_{i=1}^n (Y_i - \hat{Y_i})^2  = \frac{1}{n} \sum_{i=1}^n (Y_i - \frac{a}{ 1+ \exp\{-b(T_i-c)\}})^2
\] Then our goal is to minimize the above loss function with respect to
\((a,b,c)\). To do this, we can apply Newton-Raphson algorithm. Let
\(\boldsymbol{\theta} = (a,b,c)\). Newton's method suggests to update
\(\boldsymbol{\theta}\) iteratively, such that the \(i\)th step is given
by \[
\boldsymbol{\theta}_i = \boldsymbol{\theta}_{i-1} -[\nabla^2 l(\boldsymbol{\theta}_{i-1})]^{-1} \nabla l(\boldsymbol{\theta}_{i-1})
\] where \(\nabla l(\boldsymbol{\theta}_{i-1})\) is the gradient, and
\([\nabla^2 l(\boldsymbol{\theta}_{i-1})]^{-1}\) is the Hessian matrix.

In this particular case, we replace the Hessian matrix with an identity
matrix to simplify the computation and increase the efficiency.
Step-halfing is also incorporated to control the step size to make sure
we have a monotone trend.. Then the \(i\)th step of our Newton algorithm
is given by

\[
\boldsymbol{\theta}_i = \boldsymbol{\theta}_{i-1} + \lambda \boldsymbol H_{i-1, p \times p} \nabla l(\boldsymbol{\theta}_{i-1})
\] where \(\boldsymbol H_{i-1, p \times p} = I_{p \times p}\),
\(\lambda \in (0,1)\).

The gradient vector \(\nabla l(\boldsymbol{\theta}_{i-1})\) of our loss
function is: \begin{equation}
\begin{pmatrix}
  {\partial l}/{\partial a}\\
  {\partial l}/{\partial b}\\
  {\partial l}/{\partial c}
\end{pmatrix}
=
\begin{pmatrix}
  \frac{2}{n}\sum_{i=1}^n (Y_i - \frac{a}{ 1+ \exp \{-b(T_i-c)\}}) \cdot \frac{-1}{1+\exp \{-b(T_i-c)\} }\\
  \frac{2}{n}\sum_{i=1}^n (Y_i - \frac{a}{ 1+ \exp \{-b(T_i-c)\}}) \cdot \frac{-a(T_i-c) \exp\{-b(T_i-c)\})}{(1+\exp \{-b(T_i-c)\})^2} \\
  \frac{2}{n}\sum_{i=1}^n (Y_i - \frac{a}{ 1+ \exp \{-b(T_i-c)\}}) \cdot \frac{ab \exp\{-b(T_i-c)\})}{(1+\exp \{-b(T_i-c)\})^2}
\end{pmatrix}
\end{equation}

We set the convergence criteria to be that the difference in MSE of two
consecutive iterations is smaller than \(10^{-5}\), and the maximum
iteration number to be 1000.

\hypertarget{clustering}{%
\subsection{3.2 Clustering}\label{clustering}}

To understand which countries/regions are similar in terms of the
trajectory of COVID-19 cases, we applied two clustering methods, K-means
and Guassian mixture model, to group the fitted parameters
\((\hat{a},\hat{b},\hat{c})\).

The guassian mixture model assumes that
\(\{\mathbf x_1,\mathbf x_2,...,\mathbf x_n \} \in \mathbb R^p\) are
i.i.d. random vectors following a mixture mulitvariate normal
distributions with \(k\) hidden groups. In this case,
\(\mathbf x_i = (Y_i, T_i)'\) and \((a,b,c) \in \mathbb R^3\). And

\[\mathbf x_i\sim
\begin{cases}
N(\boldsymbol \mu_1, \Sigma_1), \mbox{with probability }p_1 \\
N(\boldsymbol \mu_2, \Sigma_2), \mbox{with probability }p_2\\
\quad\quad\vdots\quad\quad,\quad\quad \vdots\\
N(\boldsymbol \mu_k, \Sigma_k), \mbox{with probability }p_k\\
\end{cases}
\]

\[\mathbf x_i\sim
\begin{cases}
N(\boldsymbol \mu_1, \Sigma_1), \mbox{with probability }p_1 \\
N(\boldsymbol \mu_2, \Sigma_2), \mbox{with probability }p_2\\
\quad\quad\vdots\quad\quad,\quad\quad \vdots\\
N(\boldsymbol \mu_k, \Sigma_k), \mbox{with probability }p_k\\
\end{cases}
\]

The density of a multivariate normal \(\mathbf x_i\) is
\[f(\mathbf x; \boldsymbol \mu, \Sigma) = \frac{\text{exp} (-\frac{1}{2}(\boldsymbol x - \boldsymbol \mu)^T\Sigma^{-1}(\boldsymbol x - \boldsymbol \mu))}{\sqrt{{(2\pi)^p|\Sigma|}}}\]
The observed likelihood of
\(\{\mathbf x_1,\mathbf x_2,...,\mathbf x_n \}\) is

\[L(\theta; \mathbf x_1,\mathbf x_2,...,\mathbf x_n) = \prod_{i=1}^n \sum_{j = 1}^k p_j f(\mathbf x_i; \boldsymbol \mu_j, \Sigma_j)\]
Let \(\mathbf r_i = (r_{i,1},...,r_{i,k})\in \mathbb R^k\) as the
cluster indicator of \(\mathbf x_i\), which takes form
\((0, 0,...,0,1,0,0)\) with
\(r_{i,j} =I\{ \mathbf x_i\mbox{ belongs to cluster } j\}\). The cluster
indicator \(\mathbf r_i\) is a latent variable that cannot be observed.
Therefore, we use EM algorithm to iteratively estimate model parameters.

\textbf{E-step}: Evaluate the responsibilities using the current
parameter values

\[\gamma_{i, k} ^{(t)}= P(r_{i,k}=1 |\mathbf x_i,  \theta^{(t)}) =  
\frac{p_k^{(t)}f(\mathbf x_i|\boldsymbol \mu_k^{(t)}, \Sigma_k^{(t)})}
{\sum_{j=1}^K p_k^{(t)} f(\mathbf x_i|\boldsymbol \mu_j^{(t)}, \Sigma_j^{(t)})}\]

\textbf{M-step}:\\
\(\theta^{(t+1)} = \arg\max\ell( \mathbf{x}, \mathbf{\gamma}^{(t)}, \theta )\).\\
Let \(n_k = \sum_{i=1}^n \gamma_{i, k}\), we have

\[\boldsymbol \mu_k^{(t+1)} = \frac{1}{n_k} \sum_{i=1}^n \gamma_{i, k} \mathbf x_i\]

\[\Sigma_k^{(t+1)} = \frac{1}{n_k} \sum_{i=1}^n \gamma_{i, k} (\mathbf x_i - \boldsymbol \mu_k^{(t+1)})(\mathbf x_i - \boldsymbol \mu_k^{(t+1)})^T\]

\[p_k^{(t+1)} = \frac{n_k}{n}\]

By iteratively updating the E-step and the M-step, we can reach a
solution upon convergence.

On the other hand, the \(K\)-means algorithm essentially finds cluster
centers and cluster assignments that minimize the objective function
\[J(\mathbf r, \boldsymbol \mu) = \sum_{i=1}^n\sum_{j=1}^kr_{i,j}\|\mathbf x_i-\mu_k\|^2\]
where \(\{\boldsymbol \mu_1, \boldsymbol \mu_2,...,\boldsymbol \mu_k\}\)
are the centers of the \(k\) (unknown) clusters, and
\(\mathbf r_i = (r_{i,1},...,r_{i,k})\in \mathbb R^k\) as the
\emph{hard} cluster assignment of \(\mathbf x_i\). Both Gaussian mixture
model and k-means are popular clutsering methods and we would like to
compare thier performance in our dataset.

\hypertarget{results}{%
\section{4. Results}\label{results}}

\hypertarget{logistic-curve}{%
\subsection{4.1 Logistic curve}\label{logistic-curve}}

Logistic curves were used to fit cumulative confirmed cases in each
country/region. The fitted results for some selected countries are shown
in \textbf{Figure 1}. According to the model, as of 04/29/2020, 37
regions have passed the midpoint, among which 100 regions are close to
the end of virus spreading. If the confirmed cases in a region is
greater than 95\% of the estimated \texttt{a} (upper bound) in the
logistic curve, we define the region as close to the end of virus
spreading.

In general, Regions with small population size or at an early stage of
spreading tend to have a more flat rate of growth, such as Bhutan and
Venezuela (\textbf{Figure 2}). On the other hand, some regions that are
close or have reached the end of virus spreading (at least according to
the data, \textbf{Figure 3}) tend to have a more steep rate of growth.

\hypertarget{task-1.2}{%
\subsection{Task 1.2}\label{task-1.2}}

\emph{Do you think if the logistic curve is a reasonable model for
fitting the curmulative casesa and predicting future new cases?}

I think logistic curve might not be an appropriate model for fitting the
curmulative cases and predicting future new cases. First our Newton's
algorithm suggests that although we spent much efforts in optimizing
this optimization algorithm (including standardization, replacing
Hessian by Identity mattrix and step-halfing), the convergence speed
might still have some unknown internal issue and the results highly
depending on the initial starting value if we do not do standardization,
which means that our model is not that robust and the estimated curve
shape parameters might be questionable in some unusual case. Second,
some African countries' results are little bit strange base on our
model, which might be due to too liitle data available. Finally, I may
suggest using some other exponential-distribution based curves to fit
the data and incorporate more shape paramters in the model fitting
process. Or we could some hybrid approaches to fit different types of
curves based on preclassification of countries.

\newpage

\hypertarget{figures}{%
\section{Figures}\label{figures}}

\includegraphics{report_files/figure-latex/unnamed-chunk-2-1.pdf}

\center \textbf{Figure 1} Logistic curves fitted on cumulative confirmed
cases in selected countries \center

\newpage

\includegraphics{report_files/figure-latex/unnamed-chunk-3-1.pdf}
\center \textbf{Figure 2} Logistic curves of selected countries with
flat rate of growth \center \newpage

\includegraphics{report_files/figure-latex/unnamed-chunk-4-1.pdf}
\center \textbf{Figure 3} Logistic curves of selected countries with
steep rate of growth \center \newpage

\hypertarget{task-2-clustering-your-fitted-curves}{%
\subsection{Task 2: Clustering your fitted
curves}\label{task-2-clustering-your-fitted-curves}}

\emph{Apply K-mean and Guassian mixture model (with EM algorithm) to
cluster the fitted parameters \((\hat{a},\hat{b},\hat{c})\); Which
algoirhtm does a better job in clustering those curves? Are the
resulting clusters related to geogrpahic regions, or the starting
timeing of the local virus speading, or the resources of the regions?
You may use external informations to help understand the clusters,
i.e.~find plausible explanations why some regions have similar
\((a, b,c)\)?}

\end{document}
